%Version 3 October 2023
% See section 11 of the User Manual for version history
%
%%%%%%%%%%%%%%%%%%%%%%%%%%%%%%%%%%%%%%%%%%%%%%%%%%%%%%%%%%%%%%%%%%%%%%
%%                                                                 %%
%% Please do not use \input{...} to include other tex files.       %%
%% Submit your LaTeX manuscript as one .tex document.              %%
%%                                                                 %%
%% All additional figures and files should be attached             %%
%% separately and not embedded in the \TeX\ document itself.       %%
%%                                                                 %%
%%%%%%%%%%%%%%%%%%%%%%%%%%%%%%%%%%%%%%%%%%%%%%%%%%%%%%%%%%%%%%%%%%%%%

%%\documentclass[referee,sn-basic]{sn-jnl}% referee option is meant for double line spacing

%%=======================================================%%
%% to print line numbers in the margin use lineno option %%
%%=======================================================%%

%%\documentclass[lineno,sn-basic]{sn-jnl}% Basic Springer Nature Reference Style/Chemistry Reference Style

%%======================================================%%
%% to compile with pdflatex/xelatex use pdflatex option %%
%%======================================================%%

%%\documentclass[pdflatex,sn-basic]{sn-jnl}% Basic Springer Nature Reference Style/Chemistry Reference Style


%%Note: the following reference styles support Namedate and Numbered referencing. By default the style follows the most common style. To switch between the options you can add or remove �Numbered� in the optional parenthesis. 
%%The option is available for: sn-basic.bst, sn-vancouver.bst, sn-chicago.bst%  
 
%%\documentclass[sn-nature]{sn-jnl}% Style for submissions to Nature Portfolio journals
%%\documentclass[sn-basic]{sn-jnl}% Basic Springer Nature Reference Style/Chemistry Reference Style
\documentclass[sn-mathphys-num]{sn-jnl}% Math and Physical Sciences Numbered Reference Style 
%%\documentclass[sn-mathphys-ay]{sn-jnl}% Math and Physical Sciences Author Year Reference Style
%%\documentclass[sn-aps]{sn-jnl}% American Physical Society (APS) Reference Style
%%\documentclass[sn-vancouver,Numbered]{sn-jnl}% Vancouver Reference Style
%%\documentclass[sn-apa]{sn-jnl}% APA Reference Style 
%%\documentclass[sn-chicago]{sn-jnl}% Chicago-based Humanities Reference Style

%%%% Standard Packages
%%<additional latex packages if required can be included here>

\usepackage{graphicx}%
\usepackage{multirow}%
\usepackage{amsmath,amssymb,amsfonts}%
\usepackage{amsthm}%
\usepackage{mathrsfs}%
\usepackage{xcolor}%
\usepackage{textcomp}%
\usepackage{manyfoot}%
\usepackage{booktabs}%
\usepackage{algorithm}%
\usepackage{algorithmicx}%
\usepackage{algpseudocode}%
\usepackage{listings}%
%%%%

%%%%%=============================================================================%%%%
%%%%  Remarks: This template is provided to aid authors with the preparation
%%%%  of original research articles intended for submission to journals published 
%%%%  by Springer Nature. The guidance has been prepared in partnership with 
%%%%  production teams to conform to Springer Nature technical requirements. 
%%%%  Editorial and presentation requirements differ among journal portfolios and 
%%%%  research disciplines. You may find sections in this template are irrelevant 
%%%%  to your work and are empowered to omit any such section if allowed by the 
%%%%  journal you intend to submit to. The submission guidelines and policies 
%%%%  of the journal take precedence. A detailed User Manual is available in the 
%%%%  template package for technical guidance.
%%%%%=============================================================================%%%%

%% as per the requirement new theorem styles can be included as shown below
\theoremstyle{thmstyleone}%
\newtheorem{theorem}{Theorem}%  meant for continuous numbers
%%\newtheorem{theorem}{Theorem}[section]% meant for sectionwise numbers
%% optional argument [theorem] produces theorem numbering sequence instead of independent numbers for Proposition
\newtheorem{proposition}[theorem]{Proposition}% 
%%\newtheorem{proposition}{Proposition}% to get separate numbers for theorem and proposition etc.

\theoremstyle{thmstyletwo}%
\newtheorem{example}{Example}%
\newtheorem{remark}{Remark}%

\theoremstyle{thmstylethree}%
\newtheorem{definition}{Definition}%

\makeindex
\raggedbottom
%%\unnumbered% uncomment this for unnumbered level heads

\begin{document}

\title[Article Title]{What do you like in boardgames study}
\subtitle{A natural language processing project}
%%=============================================================%%
%% GivenName	-> \fnm{Joergen W.}
%% Particle	-> \spfx{van der} -> surname prefix
%% FamilyName	-> \sur{Ploeg}
%% Suffix	-> \sfx{IV}
%% \author*[1,2]{\fnm{Joergen W.} \spfx{van der} \sur{Ploeg} 
%%  \sfx{IV}}\email{iauthor@gmail.com}
%%=============================================================%%

\author{\fnm{Jacopo} \sur{Fichera}}\email{jacopo.fichera@studenti.unimi.it}


\keywords{keyword1, Keyword2, Keyword3, Keyword4}
\maketitle

\section{Introduction}

Boardgaming has become a really popular hobby and business in the past years.

The field is very broad and the games themselves can be very different.
At the core some elements are shared. Identifying some of these, is the goal of this project.
By scrapping review from the web platform \href{https://boardgamegeek.com/}{BoardGameGeek (BGG)}
to be collected in a corpus, this project aims in identifying the following aspects
\footnote{For a deeper insight into the domain definitions take a look at: \href{https://www.goblins.net/goblinpedia}{Goblinpedia - La tana dei Goblin}}:
% todo Da qui in giu sistema
\begin{itemize}
    \item{\textbf{Luck:}} How much randomization si present in the game.
    The higher the degree of luck in a game the lower the agency power of a player.
    Most games involve some form of luck, almost every dice game, while others do not at all or limit it as much as possible (e.g. "Guards of Atlantis 2").

    \item{\textbf{Bookkeeping:}} Is mostly a negative feature of some games.
    It is the manual recording of data or execution of automatic/semi-automatic game processes.

    \item{\textbf{Downtime:}} It is the passive time in which a player has no agency over the game.

    \item{\textbf{Interaction:}} Degree of influence of one player w.r.t to the others.
    It can be be direct like trading in "Catan" or indirect like gaining valuables in "Wyrmspan".

    \item{\textbf{Bash the leader:}} Is a phenomena present in some games where in order to win players
    have to prevent the victory of whoever is in advantage at the moment.
    This characteristic is most times exploitable by players by not acting against the current leader and instead
    trying to get closer to victory themselves.
    This forces others to sacrifice their possible victory and prioritize bashing.
    A game that most times features bashing is "Root".

    \item{\textbf{Complicated vs Complex}} A game is considered complicated if it has a steep learning curve
    but after learning the rules and the basics it is not difficult to master.
    The results of ones actions are predictable and immediate.
    An example for a complicated game is "Zombicide".
    Complex games on the other hand require critical thinking in order to achieve victory.
    Those games are hard to master and a difference in skill is easily noticeable. A good example could be "Go".
\end{itemize}

The proposed problem shares significant similarities with various aspect extractions/ sentiment analysis solutions
developed on the Citysearch corpus?TODO REF. This made me believe that the problem could be reduceable to the same task
in another domain.
For this reason for I re-implemented \textit{Attention Based Aspect Extraction}  (ABAE)\cite{he-etal-2017-unsupervised}
that was proposed for that very problem.
Alongside ABAE I also studied tweaked versions of the \textit{Latent Dirichlet Allocation} (LDA).

\paragraph{Latent Dirichlet Allocation}
LDA is a topic modelling method that has also been widely used for aspect extraction under the unsupervised learning framework,
which is our case.
%%% todo controlla se ok concettualemnte generatore -> inferrer ma gneration step non fa parte di lda defintion
%%%         come soluizione solo del suo processo
It is a probabilistic model in which documents are assumed to be generated by a mixture of topics.
LDA does not directly classify documents but assigns a topic distribution to the input.
Rather than predefined, these latent topics are inferred from the corpus that is given during the model generation.
Words are not bound to a single class and can appear across multiple ones with different probability.
The topic distribution for each document is drawn from a Dirichlet prior from which it gets its name.
%%%%

LDA has shown to be quite effective before the arrival of Transformers. A limit to overcome is that aspect extraction
is more fine-grained than simple topic modelling.
To tackle this problem I decided to explore two possible solutions:
\begin{itemize}
    \item{\textit{Local-LDA}}: A commonly used tweak on LDA.

    We feed the model sentences so that the topic extraction is local.
    % todo scrivi in conclusioni che LocalLDA era underwhelming per via del fatto che le review sono al piu solitmante righe.
    \item{\textit{NOUN-LDA}}: In the opinion mining research it has been observed that % ref paper
    the main holder of information when identifying aspects are nouns.

    I tried to apply this heuristic by generating LDA on a nouns only processed dataset.
\end{itemize}

\paragraph{Attention Based Aspect Extraction}
When ABAE was introduced the developers aimed to tackle the pitfalls of LDA and is based on two core concepts: Embeddings and Attention.
% todo continua
More precisely ABAE is a form of autoencoder where the feature matrix encodes a set of aspect embeddings.
The input generated embeddings of a sentence are weighted by an attention mechanism ?ref to paper di attention?
that helps the process to focus on relevant parts of the input sentence.
The training objective of the model is to minimize the difference between the decoded input embedding, sentence reconstruction,
and the originally calculated sentence embedding.

\paragraph{}
The trickiest part of the experiment is the lack of ground truth that brings us to an unsupervised learning framework.
In order to overcome this problem we used some commonly used clustering metrics to be able to draw some sort of conclusion on which we
take a deeper look in the coming section.

\subsection{Experimental Setup and Development Environment}
All the training procedures and notebooks were ran locally.

To make full use of the GPU power, CUDA drivers were needed. The project ran on CUDA 11.8.
To implement the ABAE model Keras was used with a Pytorch backend.
This choice was pivoted by the fact that PyTorch is very popular among the research community.
Libraries and other references are listed in the GitHub repository\cite{Fichera_Muffin_vs_Chihuahua_2024}

For reproducibility purposes all seeds are set in the code.

\begin{center}
    \begin{table}
        \begin{tabular}{|c l|}
            \hline
            Component & Model \\ [0.5ex]
            \hline\hline
            GPU       & NVIDIA GeForce RTX3070Ti \\
            \hline
            CPU       & AMD Ryzen 7 5800x        \\
            \hline
            RAM       & 32 GB (2x16GB) DDR4      \\
            \hline
            OS        & Windows 11               \\
            \hline
        \end{tabular}
        \caption{Brief overview of the machine specs}
        \label{specs}

    \end{table}

\end{center}



\section{Research question and methodology}

The task for which I developed a solution falls is a classification problem.
To identify the aspects defined by the requirements we have to first retrieve a dataset.

Before beginning building a corpus was necessary.


The problem defined more rigorously.
We want to elaborate a classification model thatfsdasd given a sentence/review is able to identify one of the classes defined.
## formal definition ##

\subsection{Dataset and pre-processing}
BGG offers a simple yet effective API to scrap data from their platform.
The API lacks a direct method for listing all boardgames, reason for the existence of a dedicated game information
repository ?todo ntoa pie pagina con link?.

To tackle possible issues of the raw data various different pre-processing pipelines were designed using modular
processing components combined.
The comments scrapped can be subject to a special formatting and, while it is possible to find the country of origin
of a reviewer if he decides to share it, the language of a comment is not explicit.
Thus, all pipelines share a filter on the language of the review, removing all non english ones.

A further step in the processing pipeline which also is the only one introducing in domain knowledge introduced was relative
to the removal of "Kickstarter" relative reviews.
A good portion of the reviews on BGG do not actually give an insight on the game aspects we are inspecting,
but focus strictly on the experience and quality of service of the product coming from a popular crowdfunding platform like "Kickstarter".
To avoid having many redundant and low information records in the dataset, and so give more space to possible informative ones,
we apply the simple heuristic of filtering out any review containing some keywords related to it (e.g. "ks", "pledge").

Reviews are in every iterations split on words recognized and transformed to their lemma thanks to a pre-trained
POS tagger and processor: \textit{spacy}. To furhter reduce redundant and undesired information the pre-processing pipeline
maps game names and numbers to generic tags.
Another step all pipelines have in common is to filter out too short review as we expect to hardly learn anything new from them.

The final designed and used pipelines are:
\begin{itemize}
    \item {\textit{default}}: The default pipeline refers to ABAE specification.

    \item {\textit{NOUN}}: Takes a spin on the default pipeline by filtering all words that are not recognized as nouns
    by the used tagger as discussed in % todo ref a LDA che spiega perche allinzio

    \item {\textit{NOUN-sentence}}: Works like NOUN but at very start of the process the review is split on sentences.
    The splitting creates branches that all work and produce single entries for the processed dataset.

    \item {\textit{default-sentence}}: Variation on default like NOUN-sentence.
\end{itemize}

After running a pipeline any duplicated review is discarded to avoid having repetitions and further introducing of
bias in the dataset.
All these pipelines have been used for an initial evaluation of the models but only for the LDA hyperparameter
tuning the work progressed on more than one of them. This is simply because ABAE is very time consuming while LDA
gave us the possibility to explore more various data types.

\subsection{Metrics}
% todo
Not having a ground truth to estimate the real performance of the model on makes the pursuit of a strong metric
for model evaluation crucial.
As proposed when ABAE was presented \cite{he-etal-2017-unsupervised} a metric that has been observed to relate
well with human judgement is \textit{topic coherence}, also known as "\textit{umass}" \textit{coherence} \cite{mimno-etal-2011-optimizing}:
$$C(t;V^[(t)]) = \sum^M_{m=2} \sum^{m-1}_{l=1} \log \frac{D(v_m^{(t)}, v_l^{(t)}) + 1}{D(v_l^{(t)})} $$.

Where $D(v)$ measures the document frequency of the word type $v$ and $D(v,v')$ the co-document frequency.
The values of the metric lie in the interval $(-\infty, 0)$. Values closer to zero yield a better coherence.

To further make considerations on the results we also consider some other metrics:
% todo?

\subsection{Developed models}
Once the processed dataset generation has been completed we try the two different approaches to see the best way
to process for each task. Both starting points have been intialized with a 'standard' set of hyperparameters
to later be tuned.

\subsection{LDA}
% todo rileggi
LDA elaboration is not a computation heavy task for modern standards therefore I experiment the different
preprocessing pipelines with ease.
The goal of the project focuses on the approach rather than the results.
Considering this a hyperparameters tuning process was still performed.
Reason for this is that in order for LDA to be competitive with ABAE in terms of results we want
to be sure on the possible solutions. This might also, and will, give us some insight on the quality of the data.

As stated before working on LDA is not much time-consuming therefore we could apply K-fold CV on each
seen configurations to better assess the actual performance.
The hypermodel has various hyperparameters($K,\alpha, \eta$) but the most important of all is $K$: the number of topics.
The hyperparameters tuning procedure, like for ABAE, is done without applying advanced approaches like bayesian optimization.
We simply applied a random search heuristic knowing that, for enough configurations, it outperforms grid search generally.

The best found configuration is then run on the full data and evaluated according to our metrics.
% todo come top
The top words for each topic, or aspect in our application, are extracted from each topic to match it a possible gold standard of the task we are working on.
This mapping is wrapped in a class that acts as multi-aspect classifier for a given text.

\subsection{ABAE}
For ABAE complexity rises.
The model is buildable on a custom set of embedding vectors which we trained on the corpus.
The embeddings model we use is an implementation of Word2Vec and works on the default parameters defined by the \textit{gensim} library.
% todo spiego che é compost da embedding -> attention -> autoencoder -> maxmargin?
To train ABAE, as proposed by the original paper, we use \textit{max margin loss}.
A metric that measures the distance between the reconstructed sample by the autoencoder and the negative samples
provided for one training iteration. It is defined as:
$$ J(\theta) = \sum_{s\in D}\sum_{i=1}^m \max(0, 1 - r_sz_s + r_sn_i)$$

The size of the negative samples for the loss computation was fixed to 20.

As for optimizer \textit{adam} has been chosen as for the original ABAE proposal.
Although \textit{SGD} is less bloated and can perform as well if not better \cite{wilson2018marginal} the gain on the
convergence speed is enough to keep this setting.

The first run on the untun                                 ed model was on default setting of the ABAE paper.
Given that the task is similar, but we cannot assume the same for the data we have, for better results
to compare with LDA a minimum of hyperparameter tuning was performed.
Unlike LDA, for time constraint reasons,  applying k-fold CV was not possible.
The dataset should be big enough to opt for the classic cross validation approach.
The tuned hyperparameters are:
\begin{itemize}
    \item {Learning rate:}
    \item {Epochs:}
    \item {Batch size:}
    \item {Learning rate:}
    \item {Embedding size:}
    \item {Aspect size:}
\end{itemize}

%% todo riscirvi questo sotto
The best found configuration is then run on the full data and evaluated according to our metrics.
% todo come top
The top words for each topic, or aspect in our application, are extracted from each topic to match it a possible gold standard of the task we are working on.
This mapping is wrapped in a class that acts as multi-aspect classifier for a given text.
\section{Experimental results}
The first performed runs without tuning the hyperparameters gave us a broad idea
of the unoptimized solution quality on the evaluated metrics.

Both LDA and ABAE only had a small boost in performance by tuning the hyperparameters.
The best identified configurations were then used to make a final evaluation.
The different models were compared where possible.

\paragraph{LDA}
At first an additional processing pipeline was applied for LDA which splits sentences and filters them to have nouns only.
The approach was dropped as the loss of information was too high and the generated model was not on par with the others.

Hyperparameter tuning was therefore done on the two datasets: \textit{NOUN-only} and \textit{sentence}
with final best found $K=7$ for both.
As expected by the decreasing complexity of the dataset, the noun models perform better in terms of coherence.
The NOUN model was unable to recognize some key requirements this probably given by the limited number of aspects of the final configuration.

The sentence model also resulted under-segmented and did not align with the requirements.
For this reason a higher promising value from hyperparameter optimization of $K=13$ was selected
to see if the solution could be improved.
The new model indeed outperformed the best expected model with $K=7$ in both \textit{topic coherence} and \textit{perplexity}.


\begin{center}

    % todo fai questo
    \begin{table}
        \begin{tabular}{c l c}
            \hline
            Inferred Aspect   & Top relative words                                              & Gold Aspect \\ [0.5ex]
            \hline\hline
            Low-complexity    & \textit{accessible, approachable, light, casual}                &                     \\
            Riddles/Puzzles   & \textit{puzzle, riddle, logic, deductive, reasoning}            & Complex/Complicated \\
            Convoluted        & \textit{tedious, convoluted, punish, predictable, chaotic}      &                     \\
            \hline
            Frustration       & \textit{ready, frustrate, play, stop}                           & Downtime            \\
            \hline
            Instructions      & \textit{instruction, manual, answer, phrase, interpretation}    & Bookkeeping         \\
            \hline
            Interaction       & \textit{player, potentially, opponent, investor, passive, vote} &                     \\
            Cooperation       & \textit{coop, gm, competitively}                                & Interaction         \\
            Bluffing/Plotting & \textit{speculation, bluffing, partnership, auction}            &                     \\
            \hline
            Attack/Protect    & \textit{invade, protect, defend}                                & Bash the leader     \\
            \hline
            -                 & \textit{no aspect matched}                                      & Luck                \\
            \hline
            Various           & ...                                                             & Misc.               \\
            \hline
        \end{tabular}
        \caption{Gold inferred aspects on the final NOUN-LDA ($K =13$) model trained on the full data (~310k records).
        One can observe that for \textit{luck} no relation was found.
        The aspect seems to be hardly separable from the others.
        The various mapped to "Misc" are not reported but can be looked up in the repository.
        }
        \label{nounlda}

    \end{table}

\end{center}

% todo devi fare coso
For LDA the best processing pipeline in terms of result did not yield the most interpretable model in fact,
the sentence one while performing worse on the measured metrics during human inspection it seemed to be more valuable.

\paragraph{ABAE}
Experiments on ABAE were performed on both the noun and default generation pipelines.
Initially they were done on a small subset of the dataset.

Hyperparameter tuning was performed on 20 different configurations on the full dataset with some performing generally better.
The best settings were chosen by trading off the loss and measured coherence prioritizing lower coherence.
\begin{center}
    \begin{table}
        \begin{tabular}{c l c}
            \hline
            Inferred Aspect   & Top relative words                                              & Gold Aspect \\ [0.5ex]
            \hline\hline
            Low-complexity    & \textit{accessible, approachable, light, casual}                &                     \\
            Riddles/Puzzles   & \textit{puzzle, riddle, logic, deductive, reasoning}            & Complex/Complicated \\
            Convoluted        & \textit{tedious, convoluted, punish, predictable, chaotic}      &                     \\
            \hline
            Frustration       & \textit{ready, frustrate, play, stop}                           & Downtime            \\
            \hline
            Instructions      & \textit{instruction, manual, answer, phrase, interpretation}    & Bookkeeping         \\
            \hline
            Interaction       & \textit{player, potentially, opponent, investor, passive, vote} &                     \\
            Cooperation       & \textit{coop, gm, competitively}                                & Interaction         \\
            Bluffing/Plotting & \textit{speculation, bluffing, partnership, auction}            &                     \\
            \hline
            Attack/Protect    & \textit{invade, protect, defend}                                & Bash the leader     \\
            \hline
            -                 & \textit{no aspect matched}                                      & Luck                \\
            \hline
            Various           & ...                                                             & Misc.               \\
            \hline
        \end{tabular}
        \caption{Gold inferred aspects on the final ABAE model trained on the full data (~310k records).
        One can observe that for \textit{luck} no relation was found.
        The aspect seems to be hardly separable from the others.
        The various mapped to "Misc" are not reported but can be looked up in the repository.
        }
        \label{best-310}

    \end{table}

\end{center}

\begin{center}
    \begin{table}
        \begin{tabular}{c l c}
            \hline
            Inferred Aspect    & Top relative words                                     & Gold Aspect \\ [0.5ex]
            \hline\hline
            Strategy-Asymmetry & \textit{tactic, layer, tactical, strategic, asymetric} & Complex/Complicated \\
            Weight             & \textit{weight, playtime, length, long}                &                     \\
            \hline
            Frustration        & \textit{tend, frustrating, annyoying, drag, problem}   & Downtime            \\
            Analysis Paralysis & \textit{decision, choice, planning, paralzsis}         &                     \\
            \hline
            Game mechanisms    & \textit{scenario, progression, app, ai}                & Bookkeeping         \\
            Ruleset            & \textit{rule, explain, teach, learn, rulset}           &                     \\
            \hline
            Cooperation        & \textit{cooperative, coop, party, family}              & Interaction         \\
            \hline
            Player blocking    & \textit{opponent, force, block, avoid}                 & Bash the leader     \\
            \hline
            Cards/Dice         & \textit{card, flip, face, dice, randmolu}              & Luck                \\
            \hline
            Various            & ...                                                    & Misc.               \\
            \hline
        \end{tabular}
        \caption{Gold inferred aspects on the final ABAE model trained on a subsample of the data (~80k records).
        It is more balanced comapred to the other observed model that more heavily relies on compelxity and identifies
        more non relevant aspects for the problem to currently solve.
        The various mapped to "Misc" are not reported but can be looked up in the repository.
        }
        \label{best-80}
    \end{table}

\end{center}
% todo gold inferred by best ABAE and best LDA


\begin{center}
    \begin{table}
        \begin{tabular}{c r r r r r}
            \hline
            Model      & $\overline{C}$ & $\overline{C}_5$ & $\overline{C'}_5$ & $l$  & Perplexity\\ [0.5ex]
            \hline
            ABAE       & -12.62         & -10.65           & -10.16            & 3.98 & /          \\
            \hline
            ABAE-small & -6.72          & -5.49            & -3.63             & 4.06 & /          \\
            \hline % todo valori
            NOUN-LDA   & -2.71          & -2.34            & -2.39             & /    & 6.99       \\
            \hline  % todo valori
            sent-LDA   & 3.0            & 2.1              & /                 & -3   & n.d.       \\
            \hline
        \end{tabular}
        \caption{Evaluation results. All evalauted on the same test setw ith $C'$ being the coherence only in relevant aspects.
        }
        \label{performance-review}

    \end{table}

\end{center}

The best overall configuration was indeed the lowest in loss but, compared to the reduced
dataset version, it performed way worse in coherence metrics.
To further investigate the "80k" version of the base model was evaluated on the bigger test set.
It could be supposed that, not only there was bias for some identified aspects, but that the extended
data allowed the model to recognize less prominent patterns.

% todo rileggi
A last model on the optimal settings but trained on a downsized dataset was run.
The mapping between identified aspects and gold ones are reported in table \#\ref{best-80}.
Coherence is overall lower and the aspects seem to be stronger than the one identified by the full data model.

\paragraph{}
Unlike what expected by the ABAE paper LDA outperforms in measured metrics ABAE.
By looking at the found aspects it seems like the neural model is better at capturing more complex relations.

Despite giving a better expected performance the LDA models' mapped aspects were not
as convincing during human evaluation and unable.
They also had a hard time recognizing some required aspects.
This could be related to the loss of contextual information that relates well to
some aspects like downtime where most times it is referred as a frustrating activity
often associated with a negative adjective value thing that is lost by the NOUN only approach.

The task itself could be hard to solve as some aspects tend to overlap like downtime that is most
often related to bookkeeping.
Final models evaluation metrics of the dataset are reported in table \ref{performance-review}.

\section{Concluding remarks}

The quality of the generated models in performance perspective aren't possibly known without a test set.

By our measures the coherence is good enough.

Some possible issues in the approach were most probably in the processing of the data.
At posterior analysis maybe mapping the numbers all to a single num tag might have hurt the model.
This could mean that the processing pipeline may be too aggressive and degraded the data.


If not the processing pipeline the real problem could be the dataset.
The scrapped dataset from BGG is assumed to contain a good portion of reviews citing the searched aspects but
we have no guarantee of it.
This could be improved by applying some simple heuristic when dowanloading the reviews.
Reviews might be too similar and therefore measure the quality of it could also be a step to improve the results.

Another idea that could be applied is that to run the found model to filter out reviews that too harshly rely
on identified aspects that do not fit the requirmentns, e.g. "Game components".
This way we would be using a first, more general, model that allows us to specialize our dataset better.
From the new dataset a new model could be studied that probably would achieve better quality measured by the metric.

Ultimatively to really draw satisfying conclusions a test set should be developed but more work in the
processing pipeline and/or the dataset composition should help to solve the met issues.
Exploting information from BGG could be an alternative way to tackle this like for example using the complexity rating of a game.
We could suppose that games with high or low complexity are most likely to cite this caracteristic in reviews.
Other ideas like this could be studied.
In the end the quality of the dataset should have been better took care.

For sure both the data and the approach could be improved in order to make a more valid solution.

%% todo vedi se rispeti scaletta
data is enoug but the quality and refinement has to be increased

better identification between various review types
using data from the boardgame is anyways not a feasible approach as these topics should
be cited in any game that has a value in the spectrum (luck is cited both in games with no luck and with)
maybe it was applciable to complexity and take only reviews from games with a very high / low complexity total rating

also bgg is biased towards complex games an the most popular are those with a huge
attention requirment spn

nota che molte reviews fanno riferiemnto alla qualita dei componenti essendo piattaforma di review
utlimatively to really draw any conclusions a test set should be developed but I'd work on
better heuristics to sample significant data exploting some caracteritics of the games.



\bibliography{sn-bibliography}% common bib file
%% if required, the content of .bbl file can be included here once bbl is generated
%%\input sn-article.bbl


\end{document}
