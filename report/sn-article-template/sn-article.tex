%Version 3 October 2023
% See section 11 of the User Manual for version history
%
%%%%%%%%%%%%%%%%%%%%%%%%%%%%%%%%%%%%%%%%%%%%%%%%%%%%%%%%%%%%%%%%%%%%%%
%%                                                                 %%
%% Please do not use \input{...} to include other tex files.       %%
%% Submit your LaTeX manuscript as one .tex document.              %%
%%                                                                 %%
%% All additional figures and files should be attached             %%
%% separately and not embedded in the \TeX\ document itself.       %%
%%                                                                 %%
%%%%%%%%%%%%%%%%%%%%%%%%%%%%%%%%%%%%%%%%%%%%%%%%%%%%%%%%%%%%%%%%%%%%%

%%\documentclass[referee,sn-basic]{sn-jnl}% referee option is meant for double line spacing

%%=======================================================%%
%% to print line numbers in the margin use lineno option %%
%%=======================================================%%

%%\documentclass[lineno,sn-basic]{sn-jnl}% Basic Springer Nature Reference Style/Chemistry Reference Style

%%======================================================%%
%% to compile with pdflatex/xelatex use pdflatex option %%
%%======================================================%%

%%\documentclass[pdflatex,sn-basic]{sn-jnl}% Basic Springer Nature Reference Style/Chemistry Reference Style


%%Note: the following reference styles support Namedate and Numbered referencing. By default the style follows the most common style. To switch between the options you can add or remove �Numbered� in the optional parenthesis. 
%%The option is available for: sn-basic.bst, sn-vancouver.bst, sn-chicago.bst%  
 
%%\documentclass[sn-nature]{sn-jnl}% Style for submissions to Nature Portfolio journals
%%\documentclass[sn-basic]{sn-jnl}% Basic Springer Nature Reference Style/Chemistry Reference Style
\documentclass[sn-mathphys-num]{sn-jnl}% Math and Physical Sciences Numbered Reference Style 
%%\documentclass[sn-mathphys-ay]{sn-jnl}% Math and Physical Sciences Author Year Reference Style
%%\documentclass[sn-aps]{sn-jnl}% American Physical Society (APS) Reference Style
%%\documentclass[sn-vancouver,Numbered]{sn-jnl}% Vancouver Reference Style
%%\documentclass[sn-apa]{sn-jnl}% APA Reference Style 
%%\documentclass[sn-chicago]{sn-jnl}% Chicago-based Humanities Reference Style

%%%% Standard Packages
%%<additional latex packages if required can be included here>

\usepackage{graphicx}%
\usepackage{multirow}%
\usepackage{amsmath,amssymb,amsfonts}%
\usepackage{amsthm}%
\usepackage{mathrsfs}%
\usepackage{xcolor}%
\usepackage{textcomp}%
\usepackage{manyfoot}%
\usepackage{booktabs}%
\usepackage{algorithm}%
\usepackage{algorithmicx}%
\usepackage{algpseudocode}%
\usepackage{listings}%
%%%%

%%%%%=============================================================================%%%%
%%%%  Remarks: This template is provided to aid authors with the preparation
%%%%  of original research articles intended for submission to journals published 
%%%%  by Springer Nature. The guidance has been prepared in partnership with 
%%%%  production teams to conform to Springer Nature technical requirements. 
%%%%  Editorial and presentation requirements differ among journal portfolios and 
%%%%  research disciplines. You may find sections in this template are irrelevant 
%%%%  to your work and are empowered to omit any such section if allowed by the 
%%%%  journal you intend to submit to. The submission guidelines and policies 
%%%%  of the journal take precedence. A detailed User Manual is available in the 
%%%%  template package for technical guidance.
%%%%%=============================================================================%%%%

%% as per the requirement new theorem styles can be included as shown below
\theoremstyle{thmstyleone}%
\newtheorem{theorem}{Theorem}%  meant for continuous numbers
%%\newtheorem{theorem}{Theorem}[section]% meant for sectionwise numbers
%% optional argument [theorem] produces theorem numbering sequence instead of independent numbers for Proposition
\newtheorem{proposition}[theorem]{Proposition}% 
%%\newtheorem{proposition}{Proposition}% to get separate numbers for theorem and proposition etc.

\theoremstyle{thmstyletwo}%
\newtheorem{example}{Example}%
\newtheorem{remark}{Remark}%

\theoremstyle{thmstylethree}%
\newtheorem{definition}{Definition}%

\makeindex
\raggedbottom
%%\unnumbered% uncomment this for unnumbered level heads

\begin{document}

\title[Article Title]{What do you like in boardgames study}
\subtitle{A natural language processing project}
%%=============================================================%%
%% GivenName	-> \fnm{Joergen W.}
%% Particle	-> \spfx{van der} -> surname prefix
%% FamilyName	-> \sur{Ploeg}
%% Suffix	-> \sfx{IV}
%% \author*[1,2]{\fnm{Joergen W.} \spfx{van der} \sur{Ploeg} 
%%  \sfx{IV}}\email{iauthor@gmail.com}
%%=============================================================%%

\author{\fnm{Jacopo} \sur{Fichera}}\email{jacopo.fichera@studenti.unimi.it}
\maketitle

\section{Introduction}

Boardgames are fun and very various is many different aspects but some common elements most player take in account
are of course: [,,,] (lista puntata come breve descrizione).
The goal of this project is to extract from a corpus of reviews scrapped from BGG^ref, a web dp platform, these min grained topics.

The use of the word aspect is not a caso a s it is a common task in aspect sentiment analysis.
An aspect is [def]

The dataset generation led to a un unsupervised learning problem which has issues ... metrics bla bla..
For solving the problem I chose to approach different strategies basing on approaches seen in aspect extraction community...

\subsection{Experimental Setup}

\paragraph{Machine specs}

\subsection{LDA}
what and how

\subsection{Embedding models}
what and how

\subsection{ABAE}
what and how


\section{Research question and methodology}

The task for which I developed a solution falls is a classification problem.
To identify the aspects defined by the requirements we have to first retrieve a dataset.

Before beginning building a corpus was necessary.


The problem defined more rigorously.
We want to elaborate a classification model thatfsdasd given a sentence/review is able to identify one of the classes defined.
## formal definition ##

\subsection{Dataset and pre-processing}
BGG offers a simple yet effective API to scrap data from their platform.
The API lacks a direct method for listing all boardgames, reason for the existence of a dedicated game information
repository ?todo ntoa pie pagina con link?.

To tackle possible issues of the raw data various different pre-processing pipelines were designed using modular
processing components combined.
The comments scrapped can be subject to a special formatting and, while it is possible to find the country of origin
of a reviewer if he decides to share it, the language of a comment is not explicit.
Thus, all pipelines share a filter on the language of the review, removing all non english ones.

A further step in the processing pipeline which also is the only one introducing in domain knowledge introduced was relative
to the removal of "Kickstarter" relative reviews.
A good portion of the reviews on BGG do not actually give an insight on the game aspects we are inspecting,
but focus strictly on the experience and quality of service of the product coming from a popular crowdfunding platform like "Kickstarter".
To avoid having many redundant and low information records in the dataset, and so give more space to possible informative ones,
we apply the simple heuristic of filtering out any review containing some keywords related to it (e.g. "ks", "pledge").

Reviews are in every iterations split on words recognized and transformed to their lemma thanks to a pre-trained
POS tagger and processor: \textit{spacy}. To furhter reduce redundant and undesired information the pre-processing pipeline
maps game names and numbers to generic tags.
Another step all pipelines have in common is to filter out too short review as we expect to hardly learn anything new from them.

The final designed and used pipelines are:
\begin{itemize}
    \item {\textit{default}}: The default pipeline refers to ABAE specification.

    \item {\textit{NOUN}}: Takes a spin on the default pipeline by filtering all words that are not recognized as nouns
    by the used tagger as discussed in % todo ref a LDA che spiega perche allinzio

    \item {\textit{NOUN-sentence}}: Works like NOUN but at very start of the process the review is split on sentences.
    The splitting creates branches that all work and produce single entries for the processed dataset.

    \item {\textit{default-sentence}}: Variation on default like NOUN-sentence.
\end{itemize}

After running a pipeline any duplicated review is discarded to avoid having repetitions and further introducing of
bias in the dataset.
All these pipelines have been used for an initial evaluation of the models but only for the LDA hyperparameter
tuning the work progressed on more than one of them. This is simply because ABAE is very time consuming while LDA
gave us the possibility to explore more various data types.

\subsection{Metrics}
% todo
Not having a ground truth to estimate the real performance of the model on makes the pursuit of a strong metric
for model evaluation crucial.
As proposed when ABAE was presented \cite{he-etal-2017-unsupervised} a metric that has been observed to relate
well with human judgement is \textit{topic coherence}, also known as "\textit{umass}" \textit{coherence} \cite{mimno-etal-2011-optimizing}:
$$C(t;V^[(t)]) = \sum^M_{m=2} \sum^{m-1}_{l=1} \log \frac{D(v_m^{(t)}, v_l^{(t)}) + 1}{D(v_l^{(t)})} $$.

Where $D(v)$ measures the document frequency of the word type $v$ and $D(v,v')$ the co-document frequency.
The values of the metric lie in the interval $(-\infty, 0)$. Values closer to zero yield a better coherence.

To further make considerations on the results we also consider some other metrics:
% todo?

\subsection{Developed models}
Once the processed dataset generation has been completed we try the two different approaches to see the best way
to process for each task. Both starting points have been intialized with a 'standard' set of hyperparameters
to later be tuned.

\subsection{LDA}
% todo rileggi
LDA elaboration is not a computation heavy task for modern standards therefore I experiment the different
preprocessing pipelines with ease.
The goal of the project focuses on the approach rather than the results.
Considering this a hyperparameters tuning process was still performed.
Reason for this is that in order for LDA to be competitive with ABAE in terms of results we want
to be sure on the possible solutions. This might also, and will, give us some insight on the quality of the data.

As stated before working on LDA is not much time-consuming therefore we could apply K-fold CV on each
seen configurations to better assess the actual performance.
The hypermodel has various hyperparameters($K,\alpha, \eta$) but the most important of all is $K$: the number of topics.
The hyperparameters tuning procedure, like for ABAE, is done without applying advanced approaches like bayesian optimization.
We simply applied a random search heuristic knowing that, for enough configurations, it outperforms grid search generally.

The best found configuration is then run on the full data and evaluated according to our metrics.
% todo come top
The top words for each topic, or aspect in our application, are extracted from each topic to match it a possible gold standard of the task we are working on.
This mapping is wrapped in a class that acts as multi-aspect classifier for a given text.

\subsection{ABAE}
For ABAE complexity rises.
The model is buildable on a custom set of embedding vectors which we trained on the corpus.
The embeddings model we use is an implementation of Word2Vec and works on the default parameters defined by the \textit{gensim} library.
% todo spiego che é compost da embedding -> attention -> autoencoder -> maxmargin?
To train ABAE, as proposed by the original paper, we use \textit{max margin loss}.
A metric that measures the distance between the reconstructed sample by the autoencoder and the negative samples
provided for one training iteration. It is defined as:
$$ J(\theta) = \sum_{s\in D}\sum_{i=1}^m \max(0, 1 - r_sz_s + r_sn_i)$$

The size of the negative samples for the loss computation was fixed to 20.

As for optimizer \textit{adam} has been chosen as for the original ABAE proposal.
Although \textit{SGD} is less bloated and can perform as well if not better \cite{wilson2018marginal} the gain on the
convergence speed is enough to keep this setting.

The first run on the untun                                 ed model was on default setting of the ABAE paper.
Given that the task is similar, but we cannot assume the same for the data we have, for better results
to compare with LDA a minimum of hyperparameter tuning was performed.
Unlike LDA, for time constraint reasons,  applying k-fold CV was not possible.
The dataset should be big enough to opt for the classic cross validation approach.
The tuned hyperparameters are:
\begin{itemize}
    \item {Learning rate:}
    \item {Epochs:}
    \item {Batch size:}
    \item {Learning rate:}
    \item {Embedding size:}
    \item {Aspect size:}
\end{itemize}

%% todo riscirvi questo sotto
The best found configuration is then run on the full data and evaluated according to our metrics.
% todo come top
The top words for each topic, or aspect in our application, are extracted from each topic to match it a possible gold standard of the task we are working on.
This mapping is wrapped in a class that acts as multi-aspect classifier for a given text.
\section{Experimental results}
The first performed runs without tuning the hyperparameters gave us a broad idea
of the unoptimized solution quality on the evaluated metrics.

Both LDA and ABAE actually had only a small boost in performance by tuning the hyperparameters.
The best identified configurations were then used to make a final evaluation and try compare
the different models where possible.

\paragraph{LDA}
At first an additional processing pipeline was applied for LDA which split sentences and filtered to have nouns only.
The approach was dropped as the loss of information was too high and the generated model was not on par with the others.

Hyperparameter tuning was therefore done on the two datasets: \textit{NOUN-only} and \textit{sentence}
with final best $K$s being 11 and 7 respectively.
As expected by the decreasing complexity of the dataset, the noun models perform better in terms of coherence.

Overall coherence results are reported in the table.

% todo devi fare coso
For LDA the best processing pipeline in terms of result did not yield the most interpretable model in fact,
the sentence one while performing worse returned resulted by human judgement to be more valuable.

\paragraph{ABAE}
Experiments on ABAE were performed on both the noun and default generation pipelines.
Initially they were done on a small subset of the dataset.

Hyperparameter tuning was performed on 20 different configurations on the full dataset
with some performing generally better.
While in similar ranges of registered loss some more coherent some settings were chosen for a final model generation
prioritizing overall measured topic coherence and variance on this.
\begin{center}
    \begin{table}
        \begin{tabular}{c l c}
            \hline
            Inferred Aspect   & Top relative words                                              & Gold Aspect \\ [0.5ex]
            \hline\hline
            Low-complexity    & \textit{accessible, approachable, light, casual}                &                     \\
            Riddles/Puzzles   & \textit{puzzle, riddle, logic, deductive, reasoning}            & Complex/Complicated \\
            Convoluted        & \textit{tedious, convoluted, punish, predictable, chaotic}      &                     \\
            \hline
            Frustration       & \textit{ready, frustrate, play, stop}                           & Downtime            \\
            \hline
            Instructions      & \textit{instruction, manual, answer, phrase, interpretation}    & Bookkeeping         \\
            \hline
            Interaction       & \textit{player, potentially, opponent, investor, passive, vote} &                     \\
            Cooperation       & \textit{coop, gm, competitively}                                & Interaction         \\
            Bluffing/Plotting & \textit{speculation, bluffing, partnership, auction}            &                     \\
            \hline
            Attack/Protect    & \textit{invade, protect, defend}                                & Bash the leader     \\
            \hline
            -                 & \textit{no aspect matched}                                      & Luck                \\
            \hline
            Various           & ...                                                             & Misc.               \\
            \hline
        \end{tabular}
        \caption{Gold inferred aspects on the final ABAE model trained on the full data (~310k records).
        One can observe that for \textit{luck} no relation was found.
        The aspect seems to be hardly separable from the others.
        The various mapped to "Misc" are not reported but can be looked up in the repository.
        }
        \label{best-310}

    \end{table}

\end{center}

\begin{center}
    \begin{table}
        \begin{tabular}{c l c}
            \hline
            Inferred Aspect    & Top relative words                                     & Gold Aspect \\ [0.5ex]
            \hline\hline
            Strategy-Asymmetry & \textit{tactic, layer, tactical, strategic, asymetric} & Complex/Complicated \\
            Weight             & \textit{weight, playtime, length, long}                &                     \\
            \hline
            Frustration        & \textit{tend, frustrating, annyoying, drag, problem}   & Downtime            \\
            Analysis Paralysis & \textit{decision, choice, planning, paralzsis}         &                     \\
            \hline
            Game mechanisms    & \textit{scenario, progression, app, ai}                & Bookkeeping         \\
            Ruleset            & \textit{rule, explain, teach, learn, rulset}           &                     \\
            \hline
            Cooperation        & \textit{cooperative, coop, party, family}              & Interaction         \\
            \hline
            Player blocking    & \textit{opponent, force, block, avoid}                 & Bash the leader     \\
            \hline
            Cards/Dice         & \textit{card, flip, face, dice, randmolu}              & Luck                \\
            \hline
            Various            & ...                                                    & Misc.               \\
            \hline
        \end{tabular}
        \caption{Gold inferred aspects on the final ABAE model trained on a subsample of the data (~80k records).
        It is more balanced comapred to the other observed model that more heavily relies on compelxity and identifies
        more non relevant aspects for the problem to currently solve.
        The various mapped to "Misc" are not reported but can be looked up in the repository.
        }
        \label{best-80}
    \end{table}

\end{center}



\begin{center}
    \begin{table}
        \begin{tabular}{c r r r r r}
            \hline
            Model        & $\overline{C}$ & $\overline{C}_5$ & $\overline{C'}_5$ & $l$  & Perplexity\\ [0.5ex]
            \hline
            ABAE         & -12.62         & -10.65           & -10.16            & 3.98 & /          \\
            \hline
            ABAE-small   & -6.72          & -5.49            & -3.63             & 4.06 & /          \\
            \hline % todo valori
            NOUN-LDA     & -2.71          & -2.34            & -2.39             & /    & 6.99       \\
            \hline e % todo valori
            sentence-LDA & 3.0            & 2.1              & 2                 & -3   & n.d.       \\
            \hline
        \end{tabular}
        \caption{Evaluation results. All evalauted on the same test setw ith $C'$ being the coherence only in relevant aspects.
        }
        \label{performance-review}

    \end{table}

\end{center}


The best overall configuration was indeed the lowest in loss but, compared to the reduced
dataset version, it performed way worse in coherence metrics.
To further investigate the 80k version of the base model was evaluated on the bigger test set.
Results indicate that there might be some bias towards some topics of interest in the dataset
that by increasing the dataset size get more relevant.

One could suppose that not only there was bias for some identified aspects but that the extended
data yielded allowed the model to recognize less prominent patterns.

A last model on the optimal settings but trained on a subset of data was run and the aspects
associated to gold ones as reported in the association table.
These seem to be stronger than the ones of the full dataset model.

\paragraph{}
In the same scenario proposed by the ABAE paper ABAe outperforms LDA by a large margin.
The identified aspects also better map to the searched gold standards.

Interestingly this was not true for the NOUN only version of LDA that was more coherent.
Despite giving a better expected performance the mapped aspects were not
as convincing during human evaluation and unable to recognize some required aspects.
This could be related to the loss of contextual information that relate well to
some aspects like downtime where most times it is referred as a frustrating activity
often associated with a negative adjective value thing that is lost by this approach.

Some aspects tend to overlap as well as downtime is often related to bookkeeping and
luck is a crucial element in the discussion of complexity of games.

Final classification on the dataset are reported in table \ref{performance-review}.

\section{Concluding remarks}
The results do not compare well with what hoped as the more complex solutions yield worse metric values.
Some possible issues in the approach were most probably in the processing of the data as the pipeline may be
too aggressive and degrade the structure of the information needed by ABAE.

If not the processing pipeline the real problem could be the dataset itself.
Reviews might be too similar or generally unbalanced, therefore, to measure the quality of it could
be a first step in order to understand how to improve the results.

Another note is that the scrapped dataset from BGG is assumed to contain a good portion of reviews citing
the searched aspects, but we have no guarantee of it.
In fact the platform itself is actually known to be very biased towards complex games: of the top 20 ranked
games only 3 have a weight rating below 3 ("Monopoly" for comparison has a score of \~1.62)
\footnote{Ranking was inspected as for 03/2025}

To improve the overall dataset quality additional information given by BGG should have been exploited.
A simple idea would be the one to use the complexity rating of a game hoping that high or low complexity games
are the most likely ones to cite thi characteristic in reviews.
Other ideas like this could be studied.

Another idea that could be applied is to run the found model to filter out reviews that too harshly rely
on identified aspects that do not fit the requirements, e.g. "Game components".
This way we would be using a first more general model that allows us to specialize our dataset better.
From the new dataset a new model could be studied that probably would achieve better quality measured by the metric.

An ensemble method of the multiple generated models is also an option but that would require to find a way to give more
weight to one model compared to the other and ultimately is better done with a test set.

In the end it cannot be state if the low coherence ABAE model performs worse in the identification task
without the test set.
Ultimatively the quality of the dataset should have been better took care for and studied.




\bibliography{sn-bibliography}% common bib file
%% if required, the content of .bbl file can be included here once bbl is generated
%%\input sn-article.bbl


\end{document}
