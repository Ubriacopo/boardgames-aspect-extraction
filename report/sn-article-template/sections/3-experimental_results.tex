\section{Experimental results}

Results for the project were underwhelming.
Topic coherence was overall low.
This phenomena might be given by the locality and complexity that derives from the fine-grained aspects to identify.

% Table tuning reults
For ABAE around 20 different configurations for hyperparameters were evaluated.
Some of these performed slightly better than others but, overall, there only was a minor gain in topic coherence with
w.r.t the standard configuration that was used during the first attempt.

Measured values for LDA are a little better but still not in optimal ranges for topic modelling applications.
% todo devi fare coso
For LDA the best processing pipeline in terms of result did not yield the most interpretable model in fact,
the sentence model while performing worse returned a more valuable model.
The identified gold aspects mapped to the identified ones are reported in table. % todo ref table
From the inferred aspects we see that the model has a hard time distinguishing %todo quali
which might be given by the fact that the dataset is not balanced well enough.

The reviews obtained from BGG mostly talk about complexity and components and less about % vbedi se  qualcosa
this unbalance in the data could have led to a bad identification of some required aspects.

ABAE results are reported in table.
The results are less promising in terms of coherence but the found aspects map well to the domain requested
ones.
As expected ABAE is able to better extract underlying associations that allows it to recognize more complex relationships
between potential aspects.
