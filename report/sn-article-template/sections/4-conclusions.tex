\section{Concluding remarks}
The results do not compare well with what hoped as the more complex solutions yield worse metric values.
Some possible issues in the approach were most probably in the processing of the data as the pipeline may be
too aggressive and degrade the structure of the information needed by ABAE.

If not the processing pipeline the real problem could be the dataset itself.
Reviews might be too similar or generally unbalanced, therefore, to measure the quality of it could
be a first step in order to understand how to improve the results.

Another note is that the scrapped dataset from BGG is assumed to contain a good portion of reviews citing
the searched aspects, but we have no guarantee of it.
In fact the platform itself is actually known to be very biased towards complex games: of the top 20 ranked
games only 3 have a weight rating below 3 ("Monopoly" for comparison has a score of \~1.62)
\footnote{Ranking was inspected as for 03/2025}

To improve the overall dataset quality additional information given by BGG should have been exploited.
A simple idea would be the one to use the complexity rating of a game hoping that high or low complexity games
are the most likely ones to cite thi characteristic in reviews.
Other ideas like this could be studied.

Another idea that could be applied is to run the found model to filter out reviews that too harshly rely
on identified aspects that do not fit the requirements, e.g. "Game components".
This way we would be using a first more general model that allows us to specialize our dataset better.
From the new dataset a new model could be studied that probably would achieve better quality measured by the metric.

An ensemble method of the multiple generated models is also an option but that would require to find a way to give more
weight to one model compared to the other and ultimately is better done with a test set.

A possible note on the low performance of ABAE is the mistakenly assumed range of parameters
during hyperparameter tuning as a range of $[7,20]$ was used for the aspect size.
Unlike what supposed it seems that the data is way more complex in our project than the one
used by other applications involving ABAE. By increasing the number of aspects there was indeed
a better separation of topics while also increasing overall coherence.
This would also explain why on a smaller dataset ABAE performs better, the data is less various
and some patterns that are recognizable in the big one are not so for the

In the end it cannot be state if the low coherence ABAE model performs worse in the identification task
without the test set.
Ultimatively the quality of the dataset should have been better took care for and studied.

