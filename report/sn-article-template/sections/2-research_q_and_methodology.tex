\section{Research question and methodology}

The task for which I developed a solution falls is a classification problem.
To identify the aspects defined by the requirements we have to first retrieve a dataset.

Before beginning building a corpus was necessary.


The problem defined more rigorously.
We want to elaborate a classification model thatfsdasd given a sentence/review is able to identify one of the classes defined.
## formal definition ##

\subsection{Dataset and pre-processing}
BGG offers a simple yet effective API to scrap data from their platform.
The API lacks a direct method for listing all boardgames, reason for the existence of a dedicated game information
repository ?todo ntoa pie pagina con link?.

To tackle possible issues of the raw data various different pre-processing pipelines were designed using modular
processing components combined.
The comments scrapped can be subject to a special formatting and, while it is possible to find the country of origin
of a reviewer if he decides to share it, the language of a comment is not explicit.
Thus, all pipelines share a filter on the language of the review, removing all non english ones.

A further step in the processing pipeline which also is the only one introducing in domain knowledge introduced was relative
to the removal of "Kickstarter" relative reviews.
A good portion of the reviews on BGG do not actually give an insight on the game aspects we are inspecting,
but focus strictly on the experience and quality of service of the product coming from a popular crowdfunding platform like "Kickstarter".
To avoid having many redundant and low information records in the dataset, and so give more space to possible informative ones,
we apply the simple heuristic of filtering out any review containing some keywords related to it (e.g. "ks", "pledge").

Reviews are in every iterations split on words recognized and transformed to their lemma thanks to a pre-trained
POS tagger and processor: \textit{spacy}. To furhter reduce redundant and undesired information the pre-processing pipeline
maps game names and numbers to generic tags.
Another step all pipelines have in common is to filter out too short review as we expect to hardly learn anything new from them.

The final designed and used pipelines are:
\begin{itemize}
    \item {\textit{default}}: The default pipeline refers to ABAE specification.

    \item {\textit{NOUN}}: Takes a spin on the default pipeline by filtering all words that are not recognized as nouns
    by the used tagger as discussed in % todo ref a LDA che spiega perche allinzio

    \item {\textit{NOUN-sentence}}: Works like NOUN but at very start of the process the review is split on sentences.
    The splitting creates branches that all work and produce single entries for the processed dataset.

    \item {\textit{default-sentence}}: Variation on default like NOUN-sentence.
\end{itemize}

After running a pipeline any duplicated review is discarded to avoid having repetitions and further introducing of
bias in the dataset.
All these pipelines have been used for an initial evaluation of the models but only for the LDA hyperparameter
tuning the work progressed on more than one of them. This is simply because ABAE is very time consuming while LDA
gave us the possibility to explore more various data types.

\subsection{Metrics}
% todo
Not having a ground truth to estimate the real performance of the model on makes the pursuit of a strong metric
for model evaluation crucial.
As proposed when ABAE was presented \cite{he-etal-2017-unsupervised} a metric that has been observed to relate
well with human judgement is \textit{topic coherence}, also known as "\textit{umass}" \textit{coherence} \cite{mimno-etal-2011-optimizing}:
$$C(t;V^[(t)]) = \sum^M_{m=2} \sum^{m-1}_{l=1} \log \frac{D(v_m^{(t)}, v_l^{(t)}) + 1}{D(v_l^{(t)})} $$.

Where $D(v)$ measures the document frequency of the word type $v$ and $D(v,v')$ the co-document frequency.
The values of the metric lie in the interval $(-\infty, 0)$. Values closer to zero yield a better coherence.

To further make considerations on the results we also consider some other metrics:
% todo?

\subsection{Developed models}
Once the processed dataset generation has been completed we try the two different approaches to see the best way
to process for each task. Both starting points have been intialized with a 'standard' set of hyperparameters
to later be tuned.

\subsection{LDA}
% todo rileggi
LDA elaboration is not a computation heavy task for modern standards therefore I experiment the different
preprocessing pipelines with ease.
The goal of the project focuses on the approach rather than the results.
Considering this a hyperparameters tuning process was still performed.
Reason for this is that in order for LDA to be competitive with ABAE in terms of results we want
to be sure on the possible solutions. This might also, and will, give us some insight on the quality of the data.

As stated before working on LDA is not much time-consuming therefore we could apply K-fold CV on each
seen configurations to better assess the actual performance.
The hypermodel has various hyperparameters($K,\alpha, \eta$) but the most important of all is $K$: the number of topics.
The hyperparameters tuning procedure, like for ABAE, is done without applying advanced approaches like bayesian optimization.
We simply applied a random search heuristic knowing that, for enough configurations, it outperforms grid search generally.

The best found configuration is then run on the full data and evaluated according to our metrics.
% todo come top
The top words for each topic, or aspect in our application, are extracted from each topic to match it a possible gold standard of the task we are working on.
This mapping is wrapped in a class that acts as multi-aspect classifier for a given text.

\subsection{ABAE}
For ABAE complexity rises.
The model is buildable on a custom set of embedding vectors which we trained on the corpus.
The embeddings model we use is an implementation of Word2Vec and works on the default parameters defined by the \textit{gensim} library.
% todo spiego che é compost da embedding -> attention -> autoencoder -> maxmargin?
To train ABAE, as proposed by the original paper, we use \textit{max margin loss}.
A metric that measures the distance between the reconstructed sample by the autoencoder and the negative samples
provided for one training iteration. It is defined as:
$$ J(\theta) = \sum_{s\in D}\sum_{i=1}^m \max(0, 1 - r_sz_s + r_sn_i)$$

The size of the negative samples for the loss computation was fixed to 20.

As for optimizer \textit{adam} has been chosen as for the original ABAE proposal.
Although \textit{SGD} is less bloated and can perform as well if not better \cite{wilson2018marginal} the gain on the
convergence speed is enough to keep this setting.

The first run on the untun                                 ed model was on default setting of the ABAE paper.
Given that the task is similar, but we cannot assume the same for the data we have, for better results
to compare with LDA a minimum of hyperparameter tuning was performed.
Unlike LDA, for time constraint reasons,  applying k-fold CV was not possible.
The dataset should be big enough to opt for the classic cross validation approach.
The tuned hyperparameters are:
\begin{itemize}
    \item {Learning rate:}
    \item {Epochs:}
    \item {Batch size:}
    \item {Learning rate:}
    \item {Embedding size:}
    \item {Aspect size:}
\end{itemize}

%% todo riscirvi questo sotto
The best found configuration is then run on the full data and evaluated according to our metrics.
% todo come top
The top words for each topic, or aspect in our application, are extracted from each topic to match it a possible gold standard of the task we are working on.
This mapping is wrapped in a class that acts as multi-aspect classifier for a given text.