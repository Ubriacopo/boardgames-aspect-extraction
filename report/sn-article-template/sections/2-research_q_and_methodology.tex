\section{Research question and methodology}

The task for which I developed a solution falls is a classification problem.
To identify the aspects defined by the requirements we have to first retrieve a dataset.

Before beginning building a corpus was necessary.


The problem defined more rigorously.
We want to elaborate a classification model thatfsdasd given a sentence/review is able to identify one of the classes defined.
## formal definition ##

\subsection{Dataset and pre-processing}
BGG offers a simple yet effective API to scrap data from the platform. The API lacks only a listing method
for all boardgames, reason for the existence of a dedicated game information repository ?todo ntoa pie pagina con link?.
The comments scrapped can be subject to a special formatting and, while it is possible to find the country of origin
of a player if he decides to share it, the language of a comment is not explicit.

The scrapped dataset before any processing is available for download on: %todo link

To tackle this in the pre-processing pipeline we filter for English only comments.

Reviews are processed in different pipelines composed of the same base elements to explore the best possible representation:
\begin{itemize}
    \item {\textit{default}}: The default pipeline runs the following steps:[]
    \item {\textit{pos_noun}}:
\end{itemize}

Only in domain knowledge introduced was relative to "Kickstarter". The data downloaded are reviews and in BGG a good portion
of reviews do not actually give an insight on the game aspects we are inspecting but focus strictly on the experience
and quality of the product coming from the popular crowdfunding platform.
To avoid having many redundant and low information records in the dataset, giving more space to possible informative ones,
the simple heuristic of filtering out any review containing some keywords related to it (e.g. "ks", "pledge").

% todo tabella di review d'esempio kickstarter bad e un altra e il risultato di processing accanto
Reviews are in every iterations split on words recognized and transformed to their lemma.
Game names + dates con tag

whydsf
what
-> approaches
noun only

scaraped from BBG api
pipeline in dettaglio su dati scaricati

\subsection{Developed models}

\subsection{LDA}

\subsection{ABAE}
custom implementation of the paper
