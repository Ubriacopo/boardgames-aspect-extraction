\section{Research question and methodology}
The problem to solve is a classification one.
In order to identify the aspects defined by the requirements we have to first retrieve a dataset.

\subsection{Dataset and pre-processing}
BGG offers a simple yet effective API to scrap data from their platform.
The API lacks a direct method for listing all boardgames, reason for the existence of a dedicated game information
repository ?todo ntoa pie pagina con link?.

To tackle possible issues of the raw data, various different pre-processing pipelines were designed
using modular processing components combined.
Some of the scrapped comments were subject to special formatting that was removed by applying some regexes.
For data constraints it is desirable to work on one language only but users do not have to explicit the language of their reviews.
Thus, all pipelines share a filter on the language of the comments, removing all non english ones.

A good portion of the reviews on BGG do not actually give an insight on the game aspects we are inspecting,
but focus strictly on the experience and quality of service of the product coming from a popular crowdfunding platform like "Kickstarter".
To avoid having many redundant and low information records in the dataset, and so give more space to possible informative ones,
we apply the simple heuristic of filtering out any review containing some keywords related to it (e.g. "ks", "pledge").

Reviews are in every iteration split on recognized words and transformed to their lemma thanks to a pre-trained
POS tagger and processor: \textit{spacy}.
To further reduce redundant and undesired information the pre-processing pipeline maps game names and numbers to generic tags.
Another step all pipelines have in common is to filter out too short review as we expect to hardly learn anything new from them.

The final designed and used pipelines are:
\begin{itemize}
    \item {\textit{default}}: The default pipeline refers to ABAE specification.

    \item {\textit{NOUN}}: Takes a spin on the default pipeline by filtering all words that are not recognized as nouns
    by the used tagger. % todo ref a paere che lavora su noun only

    \item {\textit{default-sentence}}: Is a slight modification on the \texit{default} pipeline in which
    at the start of the process reviews are split on sentences by a sentence splitter (\textit{spacy}).
    These lines are considered entries of the dataset instead of the full comment.

    \item {\textit{NOUN-sentence}}: Applies the sentence splitting logic to the \texit{NOUN} pipeline.
\end{itemize}

After running a pipeline any duplicated review is discarded to avoid having repetitions and further introducing
bias in the dataset.

All the pipelines have been used for an initial generation of models but only for the LDA hyperparameter
tuning the work progressed on more than one of them. This is simply because ABAE is very time consuming while LDA is not.

\subsection{Metrics}
% todo
Not having a ground truth to estimate the real performance of the model on makes the pursuit of a strong metric
for model evaluation crucial.
As proposed when ABAE was presented \cite{he-etal-2017-unsupervised} a metric that has been observed to relate
well with human judgement is \textit{topic coherence}, also known as "\textit{umass}" \textit{coherence} \cite{mimno-etal-2011-optimizing}:
$$C(t;V^{(t)}) = \sum^M_{m=2} \sum^{m-1}_{l=1} \log \frac{D(v_m^{(t)}, v_l^{(t)}) + 1}{D(v_l^{(t)})} $$.

Where $D(v)$ is the document frequency and $D(v,v')$ the co-document frequency.
The values of the metric lie in the interval $(-\infty, 0)$ where values closer to zero represent a better coherence.
This is the only metric shared among different model architectures.

\subsection{Developed models}
Once the processed dataset generation has been completed we try the two different approaches to see the best way
to process for each task. Both starting points have been intialized with a 'standard' set of hyperparameters
to later be tuned.

The best found configuration is then run on the full data and evaluated according to our metrics.
The top words for each topic, or aspect in our application, are extracted from each topic to match it a possible
gold standard of the task we are working on.
This mapping is wrapped in a lookup class that takes the output of the model and refines it to the mapped aspects.
Values of indices mapped to the same aspect are aggregated (summed) together as the topic becomes more relevant.

\subsubsection{LDA}
% todo rileggi
LDA elaboration is not a computation heavy task for modern standards therefore I experiment the different
preprocessing pipelines with ease.
These results stand as baseline before performing hyperparameter tuning.

The hypermodel has various hyperparameters($K,\alpha, \eta$) but the most important of all is $K$: the number of topics.
The hyperparameters tuning procedure, like for ABAE, is done without applying advanced approaches like bayesian optimization.
It simply applied a random search heuristic knowing that, for enough configurations, it generally outperforms grid search.

Thanks to the low computational cost when working on LDA K-fold CV on each seen configuration could be applied.
This allowed to draw better conclusions on the various configurations and asses their actual performance.

Models were evaluated based on the coherence for some top-k words and the LDA overall perplexity.
The final model was trained on the full data and later evaluated.

\subsubsection{ABAE}
For ABAE complexity rises.

The overall structure of ABAE can be described as:
\begin{itemize}
    \item {\textbf{Embeddings}}: Generation of embeddings for each word in a \textit{bag of words} input.
    The embeddings model we use is an implementation of \texit{Word2Vec} and runs on the default parameters defined by the \textit{gensim} library.
    \item {\textbf{Attention}}: Filter the embeddings in the sentence using an attention mechanism.
    \item {\textbf{Autoencoder}}: The weighted embeddings are passed to an auto-encoder that reduces the dimensional
    space to the target \textit{aspect} size and reconstructs the attention weighted input.
    On this is then evaluated the \textit{max-margin} loss.
\end{itemize}

To train ABAE, as proposed by the original paper, we use \textit{max-margin} loss.
A metric that measures the distance between the reconstructed sample by the autoencoder and the negative samples
provided for one training iteration.
It is defined as:
$$ J(\theta) = \sum_{s\in D}\sum_{i=1}^m \max(0, 1 - r_sz_s + r_sn_i)$$

The size of the negative samples for the loss computation was fixed to $20$.

As for optimizer \textit{adam} has been chosen as for the original ABAE proposal.
Although \textit{SGD} is as effective and less bloated\cite{wilson2018marginal} the gain on
the convergence speed is enough to keep this setting.

The first run on the not tuned model was performed on the default setting of the ABAE paper.
Unlike LDA, for time constraint reasons, applying k-fold CV in the hyperparameter tuning phase was not possible.
The dataset should be big enough to opt for the classic cross validation approach.
The tuned hyperparameters are: learning rate, epochs, batch size, embedding size and aspect size.
The best found configuration is then run on the full data and evaluated according to our metrics.
Model comparison was both done on coherence and max-margin loss.