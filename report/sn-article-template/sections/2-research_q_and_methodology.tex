\section{Research question and methodology}

\subsection{Dataset and pre-processing}
BGG offers a simple yet effective API to scrap data from their platform.
It lacks a direct method for listing boardgames, these have been taken from a generated dump.

To tackle possible issues of the raw data, various different pre-processing pipelines were designed
by using modular processing components.
For data constraints it is desirable to work on one language only but users do not have to explicit the language of their reviews.
Thus, all pipelines share a filter on the language of the comments, removing all non english ones.

A good portion of the reviews on BGG do not actually give an insight on the required game aspects.
Some focus strictly on the experience and quality of service of the product coming from the popular crowdfunding platforms (e.g. "Kickstarter").
To avoid having these low information records in the dataset the simple heuristic of filtering out reviews containing the related keywords (e.g. "ks", "pledge") was applied.

Reviews were tokenized and lemmatized thanks to a pre-trained POS tagger and processor: \textit{spacy}.
To further reduce redundant and undesired information the pre-processing pipeline maps game names and numbers to generic tags.
Another step all pipelines have in common is to filter out short review as we expect to hardly learn anything new from them.

Besides a "\textit{default}" built pipeline two more were defined as:
\begin{itemize}
    \item {\textit{NOUN}}: Takes a spin on the default pipeline by filtering all words that are not recognized as nouns
    by the used tagger. % todo ref a paere che lavora su noun only

    \item {\textit{default-sentence}}: Is a slight modification on the \texit{default} pipeline in which
    at the start of the process reviews are split on sentences by a sentence splitter (\textit{spacy}).
    These lines are considered entries of the dataset instead of the full comment.
\end{itemize}

After running a pipeline duplicates are discarded to avoid increasing bias in the dataset.

\subsection{Metrics}
% todo
Not having a ground truth to estimate the real performance of the model on makes the pursuit of a strong metric
for model evaluation crucial.
As proposed when ABAE was presented \cite{he-etal-2017-unsupervised} a metric that has been observed to relate
well with human judgement is \textit{topic coherence}, also known as "\textit{umass}" \textit{coherence} \cite{mimno-etal-2011-optimizing}:
$$C(t;V^{(t)}) = \sum^M_{m=2} \sum^{m-1}_{l=1} \log \frac{D(v_m^{(t)}, v_l^{(t)}) + 1}{D(v_l^{(t)})} $$.

Where $D(v)$ is the document frequency and $D(v,v')$ the co-document frequency.
The values of the metric lie in the interval $(-\infty, 0)$ where values closer to zero represent a better coherence.
This is the only metric shared among different model architectures.

\subsection{Developed models}
The first trained models have been initialized on a "standard" set of hyperparameters.
Hyperparameter optimization by random search followed.
The best found configuration is then run on the full data and evaluated according to the selected metrics.
The top words of each found aspect are extracted and mapped by hand the gold standards where possible.

The mapping is then wrapped in a lookup class that maps the model outputs to the required aspects.
Values belonging to the same aspect are aggregated together as it becomes more relevant.

\subsubsection{LDA}
% todo rileggi
LDA elaboration is not a computation heavy task for modern standards therefore experiments on the different
preprocessing pipelines were performed with ease.
During hypermarket tuning focus went to the most important one: the number of topics $K$.
For LDA K-fold CV on each seen configuration could be applied.

Configurations were chosen based on the average coherence for some top-k words and the overall \textit{perplexity}.
The final model was trained on the full data.

\subsubsection{ABAE}
To train ABAE, as proposed by the original paper, the \textit{contrastive max-margin} loss was used.
% todo non é definizione corretta?
During the training phase a sample of sentences that acts opposed to the input sentence is needed.
This additional set of reviews are used to compute the loss after decoding is performed.
We refer to these as \textit{negative samples}.
The loss is so defined as a \textit{hinge loss} that maximizes the distance between the decoded embedding ($r_s$)
and the averaged negative samples embeddings ($n_i$) while also minimizing the distance with the attention weighted sentence embedding ($z_s$).
Being all values from the same \textit{embeddings} space the distance is measured by dot product.

More strictly the \textit{contrastive max-margin} loss:
$$J(\theta) = \sum_{s\in D}\sum_{i=1}^m \max(0, 1 - r_sz_s + r_sn_i)$$

The size of the negative samples for the loss computation was fixed to $20$.

For the optimizer \textit{adam} was chosen for two main reasons.
The first one is that it was the optimizer used in the original ABAE proposal.
Secondly, although \textit{SGD} is as effective\cite{wilson2018marginal} and less bloated
by variables, the gain in convergence speed is considerate.

The first run on the model was performed by using the default setting applied in the ABAE paper.
Instead of K-fold CV classic CV was used.
The tuned hyperparameters are: \textit{learning rate, epochs, batch size, embedding size and aspect size}.
The best found configuration is then run on the full data and evaluated according to our metrics.
Comparison between ABAE instances was done on coherence and max-margin loss.