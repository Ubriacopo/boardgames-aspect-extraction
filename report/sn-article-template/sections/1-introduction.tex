\section{Introduction}

Playing boardgames has become really popular hobby and business in the last years thanks to some history defining
games like Settlers of Catan and the later 7 Wonders.

Those games work on mechanics and differe

Root is a famous game known for its complexity.
The games rulebook is considered a law and tries to be as scrict as possible in defining the game structure.
It is also other games like Uno are so simple that the game invites you in introducing some house rules.

The field of boardgames is very broad in its different colors but at the core some elements are shared and
identifying those, is the goal of this project.
By scrapping review from the web platform BBG and creating a corpus this project aims in identifying the following aspects
^pie di pagina con scirtto for a deeper look on definitions link to goblinpedia:
\begin{itemize}
    \item[Luck] The degree of luck involved in the game.
    Most games involve some form of luck, any dice game, while other not at all like \'Guards of Atlantis 2\'.
    \item[Bookkeeping] How much a player has to keep focus and track on resources or element that involve the winning condition
    \item[Downtime] How much the player has to wait between interacting with the game between turns if there is any
    \item[Interaction] Degree of influence of one player w.r.t to the others
    \item[Bash the leader] If and when to focus on the winning player to avoid him winning.
    A game that has common situations of bashing is Root.
    \item[Complicated vs Complex] The complexity is a difficulty that does not scaled down with the skill while
    a complicated game has only a step initial learning curve.
\end{itemize}

Identifying aspects could be a task for topic modelling tasks but focused on a more finely ?parola? level.
This problem shares many similiarities with the restaurant revies one ?scrivi meglio? seen in many Aspect Extraction tasks.
In order to solve the problem I decided to experiment two different routes, one by applying LDA with minor ?accorgimenti?
and one by training an ABAE^ref model built from scratch.

The trickiest part of the experiment is the lack of ground truth that brings us to an unsupervised learning framework.
-- riscrivi
In order to overcome this problem we used some commonly used clustering metrics to be able to draw some sort of conlcusion.

\subsection{Experimental Setup and Development Environment}
To make full use of the GPU power, which is very effective for machine learning, CUDA drivers were needed.
The project ran on CUDA 11.8.
For designing the neural networks Keras was used.
Since its latest major release (3.0) the backend, which handles the calculations, is selectable.
I chose PyTorch as it is very popular among the research community.
Libraries and other references are listed in the Github repository[4].

\begin{center}
    \begin{tabular}{||c c c c||}
        \hline
        Component & Model \\ [0.5ex]
        \hline\hline
        GPU & NVIDIA GeForce RTX3070Ti (8GB VRAM) \\
        \hline
        CPU & AMD Ryzen 7 5800x 8-core processor x16 \\
        \hline
        RAM & 32 GB (2x16GB) DDR4 \\
        \hline
        OS & Windows 11 \\
        \hline
        5 & 88 & 788 & 6344 \\ [1ex]
        \hline
    \end{tabular}
\end{center}

In order to replicate at best the results obtained for the project we give a brief
on the machine used and its environments.

\paragraph{Machine specs}

\subsection{LDA}
what and how

\subsection{Embedding models}
what and how

\subsection{ABAE}
what and how

